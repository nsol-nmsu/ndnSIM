\subsection*{Requirements }

Contributions to N\+FD must be licensed under G\+PL 3.\+0 or compatible license. If you are choosing G\+PL 3.\+0, please use the following license boilerplate in all {\ttfamily .hpp} and {\ttfamily .cpp} files\+:

Include the following license boilerplate into all {\ttfamily .hpp} and {\ttfamily .cpp} files\+: \begin{DoxyVerb}/* -*- Mode:C++; c-file-style:"gnu"; indent-tabs-mode:nil; -*- */
/**
 * Copyright (c) [Year(s)],  [Copyright Holder(s)].
 *
 * This file is part of NFD (Named Data Networking Forwarding Daemon).
 * See AUTHORS.md for complete list of NFD authors and contributors.
 *
 * NFD is free software: you can redistribute it and/or modify it under the terms
 * of the GNU General Public License as published by the Free Software Foundation,
 * either version 3 of the License, or (at your option) any later version.
 *
 * NFD is distributed in the hope that it will be useful, but WITHOUT ANY WARRANTY;
 * without even the implied warranty of MERCHANTABILITY or FITNESS FOR A PARTICULAR
 * PURPOSE.  See the GNU General Public License for more details.
 *
 * You should have received a copy of the GNU General Public License along with
 * NFD, e.g., in COPYING.md file.  If not, see <http://www.gnu.org/licenses/>.
 */
\end{DoxyVerb}


If you are affiliated to an N\+S\+F-\/supported N\+DN project institution, please use the \href{http://redmine.named-data.net/projects/nfd/wiki/NDN_Team_License_Boilerplate_(NFD)}{\tt N\+DN Team License Boilerplate}.

\subsection*{Recommendations }

N\+FD code is subject to N\+FD \href{http://redmine.named-data.net/projects/nfd/wiki/CodeStyle}{\tt code style}.

\subsection*{Running unit-\/tests }

To run unit tests, N\+FD needs to be configured and build with unit test support\+: \begin{DoxyVerb}./waf configure --with-tests
./waf
\end{DoxyVerb}


The simplest way to run tests, is just to run the compiled binary without any parameters\+: \begin{DoxyVerb}# Run core tests
./build/unit-tests-core

# Run  NFD daemon tests
./build/unit-tests-daemon

# Run NFD RIB management tests
./build/unit-tests-rib
\end{DoxyVerb}


However, \href{http://www.boost.org/doc/libs/1_48_0/libs/test/doc/html/}{\tt Boost.\+Test framework} is very flexible and allows a number of run-\/time customization of what tests should be run. For example, it is possible to choose to run only a specific test suite, only a specific test case within a suite, or specific test cases within specific test suites\+: \begin{DoxyVerb}# Run only TCP Face test suite of NFD daemon tests (see tests/daemon/face/tcp.cpp)
./build/unit-tests-daemon -t FaceTcp

# Run only test case EndToEnd4 from the same test suite
./build/unit-tests-daemon -t FaceTcp/EndToEnd4

# Run Basic test case from all core test suites
./build/unit-tests-core -t */Basic
\end{DoxyVerb}


By default, Boost.\+Test framework will produce verbose output only when a test case fails. If it is desired to see verbose output (result of each test assertion), add {\ttfamily -\/l all} option to {\ttfamily ./build/unit-\/tests} command. To see test progress, you can use {\ttfamily -\/l test\+\_\+suite} or {\ttfamily -\/p} to show progress bar\+: \begin{DoxyVerb}# Show report all log messages including the passed test notification
./build/unit-tests-daemon -l all

# Show test suite messages
./build/unit-tests-daemon -l test_suite

# Show nothing
./build/unit-tests-daemon -l nothing

# Show progress bar
./build/unit-tests-core -p
\end{DoxyVerb}


There are many more command line options available, information about which can be obtained either from the command line using {\ttfamily -\/-\/help} switch, or online on \href{http://www.boost.org/doc/libs/1_48_0/libs/test/doc/html/}{\tt Boost.\+Test library} website. 