Some Linux distributions, such as Ubuntu, use \href{http://upstart.ubuntu.com/}{\tt upstart} as a standard mechanism to start system daemons, monitor their health, and restart when they die.

\subsection*{Initial setup }


\begin{DoxyItemize}
\item Edit {\ttfamily nfd.\+conf} correcting paths for {\ttfamily nfd} binary, configuration and log files.
\item Copy upstart config file for N\+FD \begin{DoxyVerb}  sudo cp nfd.conf /etc/init/
\end{DoxyVerb}

\end{DoxyItemize}

\subsubsection*{Assumptions in the default scripts}


\begin{DoxyItemize}
\item {\ttfamily nfd} is installed into {\ttfamily /usr/local/bin}
\item Configuration file is {\ttfamily /usr/local/etc/ndn/nfd.conf}
\item {\ttfamily nfd} will be run as root
\item Log files will be written to {\ttfamily /usr/local/var/log/ndn} folder, which is owned by user {\ttfamily ndn}
\end{DoxyItemize}

\subsubsection*{Creating users}

If {\ttfamily ndn} user and group does not exists, they need to be manually created. \begin{DoxyVerb}# Create group `ndn`
addgroup --system ndn

# Create user `ndn`
sudo adduser --system \
             --disabled-login \
             --ingroup ndn \
             --home /nonexistent \
             --gecos "NDN User" \
             --shell /bin/false \
             ndn
\end{DoxyVerb}


\subsubsection*{Creating folders}

Folder {\ttfamily /usr/local/var/log/ndn} should be created and assigned proper user and group\+: \begin{DoxyVerb}sudo mkdir -p /usr/local/var/log/ndn
sudo chown -R ndn:ndn /usr/local/var/log/ndn
\end{DoxyVerb}


{\ttfamily H\+O\+ME} directory for {\ttfamily nfd} should be created prior to starting. This is necessary to manage unique security credentials for the deamon. \begin{DoxyVerb}# Create HOME and generate self-signed NDN certificate for nfd
sudo -s -- ' \
  mkdir -p /usr/local/var/lib/ndn/nfd/.ndn; \
  export HOME=/usr/local/var/lib/ndn/nfd; \
  ndnsec-keygen /localhost/daemons/nfd | ndnsec-install-cert -; \
'
\end{DoxyVerb}


\subsubsection*{Configuring N\+FD\textquotesingle{}s security}

N\+FD sample configuration allows anybody to create faces, add nexthops to F\+IB, and set strategy choice for namespaces. While such settings could be a good start, it is generally not a good idea to run N\+FD in this mode.

While thorough discussion about security configuration of N\+FD is outside the scope of this document, at least the following change should be done to {\ttfamily nfd.\+conf} in authorize section\+: \begin{DoxyVerb}authorizations
{
  authorize
  {
    certfile certs/localhost_daemons_nfd.ndncert
    privileges
    {
        faces
        fib
        strategy-choice
    }
  }

  authorize
  {
    certfile any
    privileges
    {
        faces
        strategy-choice
    }
  }
}
\end{DoxyVerb}


While this configuration still allows management of faces and updating strategy choice by anybody, only N\+FD\textquotesingle{}s R\+IB Manager (i.\+e., N\+FD itself) is allowed to manage F\+IB.

As the final step to make this configuration work, nfd\textquotesingle{}s self-\/signed certificate needs to be exported into {\ttfamily localhost\+\_\+daemons\+\_\+nfd.\+ndncert} file\+: \begin{DoxyVerb}sudo -s -- '\
  mkdir -p /usr/local/etc/ndn/certs || true; \
  export HOME=/usr/local/var/lib/ndn/nfd; \
  ndnsec-dump-certificate -i /localhost/daemons/nfd > \
    /usr/local/etc/ndn/certs/localhost_daemons_nfd.ndncert; \
  '
\end{DoxyVerb}


\subsection*{Enable auto-\/start }

After copying the provided upstart script, {\ttfamily nfd} daemon will automatically run after the reboot. To manually start them, use the following commands\+: \begin{DoxyVerb}sudo start nfd
\end{DoxyVerb}


\subsection*{Disable auto-\/start }

To stop {\ttfamily nfd} daemon, use the following commands\+: \begin{DoxyVerb}sudo stop nfd
\end{DoxyVerb}


Note that as long as upstart files are present in {\ttfamily /etc/init/}, the daemon will automatically start after the reboot. To permanently stop {\ttfamily nfd} daemon, delete the upstart files\+: \begin{DoxyVerb}sudo rm /etc/init/nfd.conf\end{DoxyVerb}
 