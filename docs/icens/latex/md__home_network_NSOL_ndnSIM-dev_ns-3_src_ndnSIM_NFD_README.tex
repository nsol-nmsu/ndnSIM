For complete documentation, including step-\/by-\/step installation instructions and tutorials, please visit the \href{http://named-data.net/doc/NFD/}{\tt N\+FD homepage}.

\subsection*{Overview}

N\+FD is a network forwarder that implements and evolves together with the Named Data Networking (N\+DN) \href{http://named-data.net/doc/ndn-tlv/}{\tt protocol}. After the initial release, N\+FD will become a core component of the \href{http://named-data.net/codebase/platform/}{\tt N\+DN Platform} and will follow the same release cycle.

N\+FD is an open and free software package licensed under G\+PL 3.\+0 license and is the centerpiece of our committement to making N\+DN\textquotesingle{}s core technology open and free to all Internet users and developers. For more information about the licensing details and limitation, refer to https\+://github.com/named-\/data/\+N\+F\+D/blob/master/\+C\+O\+P\+Y\+I\+N\+G.\+md \char`\"{}`\+C\+O\+P\+Y\+I\+N\+G.\+md`\char`\"{}.

N\+FD is developed by a community effort. Although the first release was mostly done by the members of \href{http://named-data.net/project/participants/}{\tt N\+S\+F-\/sponsored N\+DN project team}, it already contains significant contributions from people outside the project team (for more details, refer to https\+://github.com/named-\/data/\+N\+F\+D/blob/master/\+A\+U\+T\+H\+O\+R\+S.\+md \char`\"{}`\+A\+U\+T\+H\+O\+R\+S.\+md`\char`\"{}). We strongly encourage participation from all interested parties, since broader community support is key for N\+DN to succeed as a new Internet architecture. Bug reports and feedback are highly appreciated and can be made through \href{http://redmine.named-data.net/projects/nfd}{\tt Redmine site} and the \href{http://www.lists.cs.ucla.edu/mailman/listinfo/ndn-interest}{\tt ndn-\/interest mailing list}.

The main design goal of N\+FD is to support diverse experimentation of N\+DN technology. The design emphasizes {\itshape modularity} and {\itshape extensibility} to allow easy experiments with new protocol features, algorithms, new applications. We have not fully optimized the code for performance. The intention is that performance optimizations are one type of experiments that developers can conduct by trying out different data structures and different algorithms; over time, better implementations may emerge within the same design framework.

N\+FD will keep evolving in three aspects\+: improvement of the modularity framework, keeping up with the N\+DN protocol spec, and addition of other new features. We hope to keep the modular framework stable and lean, allowing researchers to implement and experiment with various features, some of which may eventually work into the protocol spec. 